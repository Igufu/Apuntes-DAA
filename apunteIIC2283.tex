%----------------------------------------------------------------------------------------
%	PACKAGES AND OTHER DOCUMENT CONFIGURATIONS
%----------------------------------------------------------------------------------------

\documentclass[letterpaper]{article}

\usepackage[utf8]{inputenc} % Required for inputting international characters
% \usepackage[T1]{fontenc} % Output font encoding for international characters
\usepackage[margin=1in]{geometry}
% \usepackage{mathpazo} % Palatino font
\usepackage[spanish]{babel}\decimalpoint % Soporte para español
\usepackage{graphicx} % Paquete para imágenes
\usepackage{fancyhdr} % Paquete para customización de las páginas
\usepackage{lastpage} % Saber última página
\usepackage[hidelinks]{hyperref} % Links para títulos
\usepackage{amsmath}
\usepackage{amssymb}
\usepackage{amsfonts}
\usepackage{xcolor}
\usepackage{empheq}
\usepackage{fancybox}
\usepackage[labelfont=bf]{caption}
\usepackage{array}
\usepackage{float}
\usepackage[most]{tcolorbox}
\usepackage{bookmark}
\usepackage{wrapfig}
\usepackage{multicol}
\usepackage{enumitem}
\usepackage[noend]{algorithm2e}
\usepackage{setspace}
\usepackage{stmaryrd}
\usepackage{ntheorem}
\usepackage{float}
\setlength{\algomargin}{0pt}



\definecolor{autogray}{RGB}{235,235,235}
\definecolor{lightorange}{RGB}{255,240,220}

\newtcolorbox{greenbox}{
enhanced,
boxrule=0pt,frame hidden,
borderline west={4pt}{0pt}{green!75!black},
colback=green!10!white,
sharp corners
}

\newtcolorbox{redbox}{
enhanced,
boxrule=0pt,frame hidden,
borderline west={4pt}{0pt}{red!75!black},
colback=red!10!white,
sharp corners
}

\newtcolorbox{bluebox}{
enhanced,
boxrule=0pt,frame hidden,
borderline west={4pt}{0pt}{blue!75!black},
colback=blue!10!white,
sharp corners
}

{}\newtcbtheorem[auto counter,number within=section]{ejem}%
  {Ejemplo}{
      fonttitle=\bfseries\upshape, 
    %   fontupper=\slshape,
      arc=0mm, 
      colback=autogray,
      left=4pt,
      right=4pt,
      breakable, 
      enhanced jigsaw,
      colframe=gray!75!black}
      {ejemplo}

\newtcbtheorem[auto counter]{theo}%
{Teorema}{fonttitle=\bfseries\upshape, fontupper=\slshape,
    arc=0mm, colback=white,colframe=black!85!white}{theorem}

\newtcbtheorem[auto counter,number within=section]{ejer}%
  {Ejercicio}{
      fonttitle=\bfseries\upshape,
      arc=0mm,
      coltitle=black!85!black,
      colback=orange!15!white,
      breakable,
      left=4pt,
      right=4pt,
      enhanced jigsaw,
      colframe=orange!70!white}
      {ejercicio}



\pagestyle{fancy} % Definir estilo de páginas
\fancyhf{}
\lhead{\textbf{\scriptsize\rightmark}}
\rhead{\textbf{\footnotesize\leftmark}}
\lfoot{\textbf{\footnotesize IIC2283}}
\cfoot{\textbf{\footnotesize Diseño y Análisis de Algoritmos}}
\rfoot{\textbf{\footnotesize Página: \thepage\ de \pageref{LastPage}}}
\setlength{\shadowsize}{2pt}
\setlength{\fboxsep}{5pt}
\setlength\parindent{0pt}
\setlength{\headheight}{19.77605pt}
\setlength{\columnsep}{30pt}
\addtolength{\topmargin}{-4.77605pt}
\bookmarksetup{numbered}
\setlist[itemize]{leftmargin=*}
\setlist[enumerate]{leftmargin=*}

% Renombre de comandos
\addto\captionsspanish{% Replace "english" with the language you use
  \renewcommand{\contentsname}
    {\LARGE Índice}
}
\addto\captionsspanish{% Replace "english" with the language you use
  \renewcommand{\tablename}
    {Tabla}
}
\renewcommand\arraystretch{1.5}
\renewcommand{\labelitemi}{{\raisebox{.3\height}{\scalebox{0.6}{$\blacklozenge$}}}}
\renewcommand{\headrulewidth}{0.4pt}
\renewcommand{\footrulewidth}{0.4pt}

% Nuevos comandos
\newcommand{\p}{\vspace{1em}}
\empheqset{marginbox=\psframebox}
\newcommand{\alignformula}[1]{
    \begin{empheq}[box=\shadowbox*]{align*}
        #1
    \end{empheq}
}
\newcommand{\fig}[3]{
    \begin{figure}[H]
        \centering
        \includegraphics[scale=#2]{#1}
        \caption{#3}
        \end{figure}
}
\newcommand{\img}[2]{
    \begin{figure}[H]
        \centering
        \includegraphics[scale=#2]{#1}
        \end{figure}
}
\newcommand{\enlace}[3]{\textcolor{#3}{\textbf{\href{#1}{#2}}}}
\newcommand{\ejemplo}[3]{\begin{ejem}{#1}{#2}#3\end{ejem}\vspace{1pt}}
\newcommand{\ejercicio}[3]{\begin{ejer}{#1}{#2}#3\end{ejer}\vspace{1pt}}
\newcommand{\teorema}[3]{\begin{theo}{#1}{#2}#3\end{theo}\vspace{1pt}}
\newcommand{\li}[2]{\displaystyle{\lim_{#1 \to #2}}}
\newcommand{\ca}[1]{\mathcal{#1}}
\newcommand{\mbb}[1]{\mathbf{#1}}
\newcommand{\first}{\texttt{first}}
\newcommand{\follow}{\texttt{follow}}
\newcommand{\gll}{\text{LL}}
\newcommand{\overunder}[3]{\underset{#2}{\overset{#3}{#1}} }

\newcommand\vartextvisiblespace[1][.5em]{%
  \makebox[#1]{%
    \kern.07em
    \vrule height.3ex
    \hrulefill
    \vrule height.3ex
    \kern.07em
  }% <-- don't forget this one!
}

\newcommand{\sbar}{\vartextvisiblespace}

\usepackage{etoolbox}
\apptocmd{\lim}{\limits}{}{}

\begin{document}

%----------------------------------------------------------------------------------------
%	TITLE PAGE
%----------------------------------------------------------------------------------------

\begin{titlepage} % Suppresses displaying the page number on the title page and the subsequent page counts as page 1
    \newcommand{\HRule}{\rule{\linewidth}{0.5mm}} % Defines a new command for horizontal lines, change thickness here

    \center % Centre everything on the page

    %------------------------------------------------
    %	Headings
    %------------------------------------------------

    \textsc{\LARGE Pontificia Universidad Católica de Chile}\\[1.5cm] % Main heading such as the name of your university/college

    \textsc{\Large Departamento de Ciencia de la Computación}\\[0.5cm] % Major heading such as course name

    \textsc{\large Apunte IIC2283}\\[0.5cm] % Minor heading such as course title

    %------------------------------------------------
    %	Title
    %------------------------------------------------

    \HRule\\[0.5cm]

    {\huge\bfseries Diseño y Análisis de Algoritmos}\\[0.4cm] % Title of your document

    \HRule\\[1.5cm]

    %------------------------------------------------
    %	Author(s)
    %------------------------------------------------

    \begin{minipage}{0.4\textwidth}
        \begin{flushleft}
            \large
            \textit{Autores}\\
            Sebastián \textsc{Besoain} \\% Your name
            Pedro \textsc{Salinas} \\
            Ignacio \textsc{Gutiérrez}
        \end{flushleft}
    \end{minipage}
    ~
    \begin{minipage}{0.4\textwidth}
        \begin{flushright}
            \large
            \textit{En base a clases de}\\
            Prof. Diego \textsc{Arroyuelo} \\
            Prof. Juan P. \textsc{Castillo}% Supervisor's name
        \end{flushright}
    \end{minipage}
 

    % If you don't want a supervisor, uncomment the two lines below and comment the code above
    {\large\textit{Template}}\\
    Cristóbal \textsc{Rojas} % Your name

    %------------------------------------------------
    %	Date
    %------------------------------------------------

    \vfill\vfill\vfill % Position the date 3/4 down the remaining page

    {\large\today} % Date, change the \today to a set date if you want to be precise

    %------------------------------------------------
    %	Logo
    %------------------------------------------------

    \vfill\vfill
    \includegraphics[width=0.3\textwidth]{img/logo.png}\\[1cm] % Include a department/university logo - this will require the graphicx package

    %----------------------------------------------------------------------------------------

    \vfill % Push the date up 1/4 of the remaining page

\end{titlepage}

% \thispagestyle{empty}
% \begin{redbox}
%     \textbf{Advertencia:} Este no es un documento oficial del curso Cálculo I - MAT1610, el apunte puede presentar fallas y por ende generar error en sus cálculos. Debido a esto, el autor \textbf{no} se hace responsable de cualquier respuesta errónea que pueda dar en alguna prueba, por lo que es su obligación revisar si los contenidos mostrados aquí coinciden con el material oficial del curso. De cualquier manera, como lector está invitado a reportar cualquier falla para así mejorar este apunte cada vez más. Puede enviar un correo a \href{mailto:cristobalrojas@uc.cl}{cristobalrojas@uc.cl}. ¡Muchas gracias!
% \end{redbox}

% \newpage

\thispagestyle{empty}
\pdfbookmark[section]{\contentsname}{toc}
\tableofcontents

\newpage

\include{ch1/contenido1}

\section{Análisis de la eficiencia de un algoritmo}

Para esta sección es bueno recordar la definición formal de la notación asintótica:

\definicion{}{}{
    Sea $f : \mathbb{N} \rightarrow \mathbb{R}^+$ se define la notación asintótica como los siguientes conjuntos de funciones:

    \begin{itemize}
        \item $\mathcal{O}(f) = \{ g : \mathbb{N} \rightarrow \mathbb{R}^+ \mid 
        (\exists c \in \mathbb{R}^+)(\exists n_0 \in \mathbb{N})(\forall n \geq n_0)(g(n) \leq c \cdot f(n)) \}$

        \item ${\Omega}(f) = \{ g : \mathbb{N} \rightarrow \mathbb{R}^+ \mid 
        (\exists c \in \mathbb{R}^+)(\exists n_0 \in \mathbb{N})(\forall n \geq n_0)(g(n) \geq c \cdot f(n)) \}$

        \item ${\Theta}(f) = \mathcal{O}(f) \cap {\Omega}(f)$
    \end{itemize}
}

No será de mucha utilidad en nuestro caso, pero se puede saber el rango del valor de los siguientes límites:

\teorema{}{}{
    Sea $f : \mathbb{N} \rightarrow \mathbb{R}^+$ y $g : \mathbb{N} \rightarrow \mathbb{R}^+$, luego

    \begin{itemize}
        \item Si $f(n) \in \mathcal{O}(g(n))$ entonces $\lim_{n \to \infty} \frac{f(n)}{g(n)} \in [0, \infty)$
        \item Si $f(n) \in {\Omega}(g(n))$ entonces $\lim_{n \to \infty} \frac{f(n)}{g(n)} \in (0, \infty]$
        \item Si $f(n) \in {\Theta}(g(n))$ entonces $\lim_{n \to \infty} \frac{f(n)}{g(n)} \in (0, \infty)$

    \end{itemize}
}

\subsection{Ecuaciones de recurrencia}

Una \textbf{recurrencia} es una ecuación o desigualdad que describe
una función en términos de su valor sobre inputs más pequeños.

\ejemplo{}{}{
El peor caso de \textsc{Merge-Sort} está dado por la ecuación de recurrencia

\[
    T(n) = \begin{cases}
            \Theta(1) & \text { si } n = 1\\
            T(\lceil n/2 \rceil) + T(\lfloor n/2 \rfloor) + \Theta(n) & \text{ si } n > 1
            \end{cases}
\]

}

Existen varios métodos para resolver estas recurrencias, uno de ellos es la \textbf{sustitución de variables}
que consiste en dos pasos:

\begin{enumerate}
    \item  Adivinar la forma de la solución
    \item Usar inducción para encontrar las constantes y mostrar que la solución funciona
\end{enumerate}

\ejemplo{}{}{
Considere la recurrencia:
\[
    T(n) = 2T(\lfloor n/2 \rfloor) + n
\]

Adivinamos primero que la solución es $T(n) \in \mathcal{O}(n \log n)$, el método de sustitución
requiere demostrar que $T(n) \leq cn \log n$ para algún $c > 0$. Empezamos asumiendo que se cumple 
para todo $m < n$ con $m$ positivo, en particular para $m = \lfloor n/2 \rfloor$, luego tenemos 
que $T(\lfloor n/2 \rfloor) \leq c \lfloor n/2 \rfloor \log(\lfloor n/2\rfloor)$. Sustituyendo en la 
ecuación tenemos que 
\begin{align*}
    T(n)
    &\le 2\bigl(c\lfloor n/2 \rfloor \lg(\lfloor n/2 \rfloor)\bigr) + n \\ 
    &\le cn \lg(n/2) + n \tag{\text{Notar que} $c$ \text{sigue siendo una constante}} \\
    &= cn \lg n - cn \lg 2 + n \\
    &= cn \lg n - cn + n \\
    &\le cn \lg n .
\end{align*}
Solo se necesita que $c \geq 1$.

\vspace*{10pt}

La inducción matemática ahora requiere que mostremos nuestra solución para las condiciones límite. 
Note que para $n = 1$, entonces:

\begin{align*}
    T(1) 
    &\leq c1\log 1 \\
    &= 0
\end{align*}
Lo que es falso ya que $T(1) = 1$. Aquí gracias a la definición formal podemos fijar un $n_0$ que funcione
en este caso podemos elegir $n_0 = 2$. Al escoger este $n_0$, debemos demostrar como casos 
base todos aquellos casos donde la recursión de $T(n)$ resulte en un caso menor a $n_0 = 2$ aquí serían $T(2)$ y $T(3)$ ya que ambos 
dependen de $T(1)$ directamente (dado la ecuación). Queda como ejercicio para el lector definir un $c$ y demostrar por inducción
la cota.
}

\ejercicio{}{a}{
Considere nuevamente la recurrencia de \textsc{Merge-Sort}:
\[ 
    T(n) = \begin{cases}
            \Theta(1) & \text { si } n = 1\\
            T(\lceil n/2 \rceil) + T(\lfloor n/2 \rfloor) + \Theta(n) & \text{ si } n \geq 2
            \end{cases}
\]

Encuentre el orden $f : \mathbb{N} \rightarrow R^+$ de $\mathcal{O}$ de $T(n)$ y demuestre formalmente que $T(n) \in \mathcal{O}(f)$.

\vspace*{10pt}
\color{orange!100!black} 

\color{orange!100!black} \textbf{Solución:} Escogemos $f = n \log n$, en este caso particular nos conviene el reemplazo 
$T(n) \leq cn \log (n-1)$ debido a las funciones techo en la ecuación, luego 
\begin{align*}
    T(n) 
    &\leq c \lceil n/2 - 1 \rceil \log  \lceil n/2 - 1 \rceil + c \lfloor n/2 - 1 \rfloor \log  \lfloor n/2 - 1 \rfloor + n\\
    &\leq c \lceil n/2 - 1 \rceil (2\log \lceil n/2 - 1 \rceil) \\
    &\leq cn \Biggl(\log \frac{(n+1)}{2} - 1\Biggr) \\
    &\leq cn \log (n - 1)
\end{align*}


Ahora falta encontrar $c$, $n_0$ y demostrar por inducción que se cumple para todo $n \geq n_0$. En este caso elegiremos 
$c = 3$ y $n_0 = 3$, ya que para $T(1)$ y $T(2)$ no se cumple. Luego nuestros casos bases son $\{3,4,5\}$ para los cuales
se cumple (lo puede comprobar), luego para el paso inductivo con $n \geq 6$ tenemos que:
\begin{align*}
    T(n) 
    &\leq 3 \lceil n/2 - 1 \rceil \log  \lceil n/2 - 1 \rceil + \lfloor n/2 - 1 \rfloor \log  \lfloor n/2 - 1 \rfloor + n\\
    &\leq 3n (\log \lceil n/2 - 1 \rceil) \\
    &\leq 3n \Biggl(\log \frac{(n+1)}{2} - 1\Biggr) \\
    &\leq 3n \log (n - 1) \hspace*{10pt}\triangle
\end{align*}


}


 




\include{ch2/ejercicios2}

\section{Análisis de caso promedio}
\include{ch3/ejercicios3}

\section{Técnicas para demostrar cotas inferiores}
\subsection{Ejercicios del capítulo}

\section{Técnicas fundamentales de diseño de algoritmos}
\include{ch5/ejercicios5}

\section{Transformaciones de domino}
\subsection{Representación de un polinomio}

La representación canónica de un polinomio $p(x)$ no nulo es 

\[
    p(x) = \sum_{i = 0}^{n - 1}a_ix^i
\]

Donde $n\geq 1$, $a_{n-1} \neq 0$ y el grado de $p(x)$ es $n-1$
\include{ch6/ejercicios6}

\section{Algoritmos aleatorizados}
\subsection{Algoritmos de Monte Carlo}

Un algoritmo de \textbf{Monte Carlo} es aquel que siempre entrega un resultado, pero hay una probabilidad 
de que sea incorrecto. Garantizan tiempo de ejecución de peor caso y suelen ser acompañados por un 
análisis de error. La probabilidad de error puede ser reducida ejecutando el algoritmo múltiples
veces de manera independiente.

\vspace*{10pt}


\ejemplo{}{}{
Considere el problema de encontrar un $1$ en alguna posición $i \in \{1,...,n\}$ de un arreglo de bits $B[1...n]$,
el cual contiene $n/2$ cantidad de $0$'s y $n/2$ cantidad de $1$'s, una solución tipo Monte Carlo sería la siguiente:

\begin{algorithm}[H]
    % \caption*{An algorithm with caption}\label{alg:two}
    \setstretch{1.25}
    \DontPrintSemicolon
    \SetKwFunction{FindMC}{Find\_1\_MC}
    \SetKwProg{Fn}{Function}{:}{}
    \SetKw{Check}{check}
    \Fn{\FindMC{$B[1...n], k$}}{
        $j := 0$

        \Repeat{$j = k$ \KwSty{or} $B[i] = 1$}{
            Elegir $i \in \{1, ..., n\}$ uniformemente al azar.\;
            $j := j + 1$ \;
        }

        \Return{$i$}
    }
\end{algorithm} 
}

\begin{itemize}
    \item Este algoritmo recibe un parámetro $k$, que permite limitar el tiempo de ejecución
    a $\mathcal{O}(k)$
    \item No siempre encuentra una posición que encuentra un $1$, dado que ejecuta a lo más $k$ iteraciones.
\end{itemize}

\ejercicio{}{Monte Carlo}{
Encuentre la probabilidad de error del algoritmo, para 
cualquier $\\color{orange!100!black}{k \in \mathbb{N}} $
}

\vspace*{10pt}

\noindent Note que a medida que aumentamos el parámetro $k$ la probabilidad de éxito va aumentando, esta
probabilidad está dada por la fórmula

\[
    \sum_{i=1}^{k} \frac{1}{2^i} = 1 - \biggl(\frac{1}{2}\biggr)^k
\]
\include{ch7/ejercicios7}

\section{Algoritmos en teoría de números}
\include{ch8/ejercicios8}


%----------------------------------------------------------------------------------------

\end{document}
