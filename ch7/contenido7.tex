\section{Algoritmos aleatorizados}
\subsection{Algoritmos de Monte Carlo}

Un algoritmo de \textbf{Monte Carlo} es aquel que siempre entrega un resultado, pero hay una probabilidad 
de que sea incorrecto. Garantizan tiempo de ejecución de peor caso y suelen ser acompañados por un 
análisis de error. La probabilidad de error puede ser reducida ejecutando el algoritmo múltiples
veces de manera independiente.

\vspace*{10pt}


\ejemplo{}{}{
Considere el problema de encontrar un $1$ en alguna posición $i \in \{1,...,n\}$ de un arreglo de bits $B[1...n]$,
el cual contiene $n/2$ cantidad de $0$'s y $n/2$ cantidad de $1$'s, una solución tipo Monte Carlo sería la siguiente:

\begin{algorithm}[H]
    % \caption*{An algorithm with caption}\label{alg:two}
    \setstretch{1.25}
    \DontPrintSemicolon
    \SetKwFunction{FindMC}{Find\_1\_MC}
    \SetKwProg{Fn}{Function}{:}{}
    \SetKw{Check}{check}
    \Fn{\FindMC{$B[1...n], k$}}{
        $j := 0$

        \Repeat{$j = k$ \KwSty{or} $B[i] = 1$}{
            Elegir $i \in \{1, ..., n\}$ uniformemente al azar.\;
            $j := j + 1$ \;
        }

        \Return{$i$}
    }
\end{algorithm} 
}

\begin{itemize}
    \item Este algoritmo recibe un parámetro $k$, que permite limitar el tiempo de ejecución
    a $\mathcal{O}(k)$
    \item No siempre encuentra una posición que encuentra un $1$, dado que ejecuta a lo más $k$ iteraciones.
\end{itemize}

\ejercicio{}{Monte Carlo}{
Encuentre la probabilidad de error del algoritmo, para 
cualquier $\\color{orange!100!black}{k \in \mathbb{N}} $
}

\vspace*{10pt}

\noindent Note que a medida que aumentamos el parámetro $k$ la probabilidad de éxito va aumentando, esta
probabilidad está dada por la fórmula

\[
    \sum_{i=1}^{k} \frac{1}{2^i} = 1 - \biggl(\frac{1}{2}\biggr)^k
\]